%%
%% Author: vadis
%% 15.01.18
%%

% Preamble
\documentclass[11pt]{article}

\author{Volodymyr Vadiasov}
\title{Probability Theory}

% Packages
\usepackage{a4wide}
\usepackage{amssymb}

% Document
\begin{document}

    \maketitle

    \section{Classic Theory}\label{sec:classic}
    %    %%
%% Author: vadis
%% 14.01.18
%%

% Preamble
%\documentclass[11pt]{article}

% Packages
\usepackage{a4wide}

% Document
%\begin{document}

    $\lim_{n \to \infty}
    \sum_{k=1}^n \frac{1}{k^2}
    = \frac{\pi^2}{6}$.

%\end{document}


    P(\Omega) = \{\omega_1, \ldots ,\omega_n \}


    P(A) = \frac{k}{n}

    \section{Properties}\label{sec:properties}

    P(\Omega) = \{\omega_1, \ldots ,\omega_n \}

    \begin{equation}
        P(\Omega) = 1
    \end{equation}

    \begin{equation}
        P(A) = 0 \iff A = \varnothing
    \end{equation}

    \begin{equation}
        P(A\sqcup B) = P(A) + P(B)
    \end{equation}

    \begin{equation}
        P(A\cup B) = P(A) + P(B) + P(A\cap B)
    \end{equation}

    \begin{equation}
        P(A_1\cup \ldots \cup A_k) = P(A_1) +  \ldots + P(A_k) - P(A_1\cap A_2) - P(A_1\cap A_3) \nonumber
        - \ldots \negmedspace - P(A_{k-1}\cap A_k) + \ldots + (-1)^{k-1}P(A_1\cap \ldots \cap A_k)
    \end{equation}

    \begin{equation}
        P(A_1\cup \ldots \cup A_k) \leq P(A_1) +  \ldots + P(A_k)
    \end{equation}

    \begin{equation}
        \overline{A} := \Omega \setminus A
    \end{equation}

    \begin{equation}
        P(\overline{A}) = 1 - P(A)
    \end{equation}

    \section{Conditional Probability}\label{sec:conditional_probability}

    \begin{equation}
        P(A \mid B) = \frac{P(A\cap B)}{P(B)}
    \end{equation}

    \section{The total probability formula}\label{sec:total_probability}

    \begin{equation}
        P(A) = \sum_{k=1}^m P(A \mid B_k)P(B_k)
    \end{equation}

    \section{The Bayes formula}\label{sec:bayes_formula}

    \begin{equation}
        P(B_i \mid A) = \frac{P(A\cap B_i)P(B_i)}{\sum_{j=1}^k P(A \mid B_j)P(B_j)}
    \end{equation}

\end{document}