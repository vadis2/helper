\part{}
\documentclass{exam}

% Packages
\usepackage{amsthm}

\author{Volodymyr Vadiasov}
\title{Combinatorics Fundamentals}

% Possible other definitions, configurations etc. that you need
\theoremstyle{definition}
\newtheorem{theorem}{Theorem}

% Document
\begin{document}

    \maketitle

    \section{Introduction}\label{sec:introduction}
    %    %%
%% Author: vadis
%% 14.01.18
%%

% Preamble
%\documentclass[11pt]{article}

% Packages
\usepackage{a4wide}

% Document
%\begin{document}

    $\lim_{n \to \infty}
    \sum_{k=1}^n \frac{1}{k^2}
    = \frac{\pi^2}{6}$.

%\end{document}

    \begin{equation}
        C^k_{n} = \frac{n!}{(n-k)!* k!}
    \end{equation}

    \begin{equation}
        \overline{C}^k_{n} = C^k_{n-k+1}
    \end{equation}

    \begin{equation}
        A^k_{n} = \frac{n!}{(n-k)!}
    \end{equation}

    \begin{equation}
        \overline{A}^k_{n} = n^k
    \end{equation}

    \begin{equation}
        P(n_1, ..., n_k)= \frac{n!}{n_1! \text... n_k!}
    \end{equation}

    \section{Pascal's rule} \label{sec:Pascal's rule}
    Pascal's triangle:

    1

    1 1  (first line)

    1 2 1

    1 3 3 1

    1 4 6 4 1

    \section{Dirichlet Principle} \label{sec:Dirichlet Principle}
    4 box, 5 rabbits

    \section{Binom Newton} \label{sec:Binom Newton}
    \begin{equation}
        (x+y)^n = \sum_{k=0}^n C^k_n * x^k * y^{n-k}
    \end{equation}

    \section{Properties} \label{sec:Properties}

    \begin{equation}
        C^k_n = C^{n-k}_n
    \end{equation}

    \begin{equation}
        C^k_{n} = C^k_{n-1} + C^{k-1}_{n-1}
    \end{equation}

    \begin{equation}
        \sum_{k=0}^n C^k_n = 2^n
    \end{equation}

    \begin{equation}
        \sum_{k=0}^n (C^k_{n})^2 = C^k_{2n}
    \end{equation}

    \begin{equation}
        C^n_{n+m} = \sum_{k=0}^m C^{n-1}_{n+k-1}
    \end{equation}

    \begin{equation}
        \sum_{k=0}^m m^2 = \frac{m(m+1)(2m+1)}{6}
    \end{equation}

    \begin{equation}
        C^0_n - C^1_n + C^2_n + ... + (-1)^n C^n_n = \left\{
        \begin{array}{rl}
            1 & \text{if } n = 0,\\
            0 & \text{if } n \geq 0.
        \end{array} \right.
    \end{equation}

    \section{Useful Tasks} \label{sec:Useful Tasks}
    Count of subsets from set (n items) if a power of each subset is odd (even): 2^{n-1}

    \section{Polynomial Coefficients} \label{sec:Polynomial Coefficients}


\end{document}